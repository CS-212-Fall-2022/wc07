\documentclass[a4paper]{exam}

\usepackage{amsmath}
\usepackage{geometry}
\usepackage{graphicx}
\usepackage{hyperref}
\usepackage{titling}

% Header and footer.
\pagestyle{headandfoot}
\runningheadrule
\runningfootrule
\runningheader{CS 212, Fall 2022}{WC 07: Robustness of Turing Machine}{\theauthor}
\runningfooter{}{Page \thepage\ of \numpages}{}
\firstpageheader{}{}{}

\printanswers

\title{Weekly Challenge 07: Robustness of Turing Machine}
\author{ungraded} % <=== replace with your student ID, e.g. xy012345
\date{CS 212 Nature of Computation\\Habib University\\Fall 2022}

\qformat{{\large\bf \thequestion. \thequestiontitle}\hfill}
\boxedpoints

\begin{document}
\maketitle

\begin{questions}
  
\titledquestion{Stay Put}

  The Turing machine we encountered in class has the transition function,
  \[
    delta(Q\times\Gamma)\to\; Q\times\Gamma\times\{L,R\}.
  \]

  Consider a variant of a Turing machine whose transition function is,
  \[
    \delta(Q\times\Gamma)\to\; Q\times\Gamma\times\{L,R, S\},
  \]
  where $S$ indicates that the head stays at its current location.

  Prove that this variant is no more powerful than the original, i.e. the set of languages accepted by both types of machines is the same.
  
  \begin{solution}
    % Enter your solution here.
  \end{solution}
\end{questions}
\end{document}

%%% Local Variables:
%%% mode: latex
%%% TeX-master: t
%%% End:
